


\subsection{END}



LACalle et al.'s comprehensive study on frequent attenders of emergency departments  


Definitions of frequent ED use vary from 2 visits per year to as high as 12 visits per year, depending on the goals of the study (Table 2).8, 9, 10, 11, 12, 13, 14, 15, 16, 17, 18, 19, 20, 21, 22, 23, 24, 25, 26, 27, 28, 29, 30, 31, 32 The threshold most commonly used to define frequent use, representing the sample mode across studies reviewed here, has been 4 or more visits per year. The authors of one large study rationalized their definition of 4 visits per year by demonstrating that this post hoc cutoff would yield a patient group that represented 25\% of all ED visits, a figure they believed was administratively significant and warranting expenditure of resources for intervention.27 These authors and others admit to the arbitrariness of these cutoffs, and many clinicians would likely use a higher number of visits to define “frequent use.” However, no study has shown a threshold number at which striking differences in resources, demographics, or clinical import are observed.

FA patients have no uniform definition, with the threshold for deciding which patients fit this classification varying between different countries, regions, and even particular departments. For instance Mercy University Hospital in Ireland classified high frequent attender as patients with between 13 to 30 Emergency Department visits per year, and very high frequent attenders as those with over 30 Emergency Department visits.\cite{bhroin2019profiling}

After consultation with the OOHC in question, the two criteria we chose for our research in classifying FA patients are those who contact the OOHC more than 50 times in a year (approximately once a week), which we describe as high FAs, and those that contact the OOHC more than 24 times in a year, which we simply describe as FAs. We are cognisant that these are higher than the threshold that are typically adopted by Emergency Departments in relation to admissions for defining FAs; however, contact of an OOHC usually represents little more than a phone call, and very rarely results in hospitalisation. As many of the duties of an OOHC amounts to telemedicine, a higher threshold is more appropriate for this type of organisation.

Despite the available research on frequent attenders of healthcare services, we have found only a few studies about frequent callers, but none from a telephone nursing perspective. Hildebrandt et al. (2004) studied frequent callers contacting a GP during out‐of‐office hours in a US context. A majority were women and frequent callers had three times more face‐to‐face consultations, prescribed drugs and specialist consultations compared with a control group (Hildebrandt et al. 2004). Spittal et al. (2015) investigated frequent callers to a crisis management helpline in Australia. They defined a frequent caller as a person who called more than 20 times per month. The callers were often lonely, anxious and had family problems. Their call times were longer than those of other callers. However, the frequent callers did have real health problems, which should be adequately addressed (Spittal et al. 2015). \cite{holmstrom2017frequent}

The role of telephone nurses is to assess callers’ health‐related needs and to provide self‐care advice or to refer callers to the appropriate level of care, which also includes a triage function and education (Kaminsky et al. 2009). This two‐fold role consists of both caring for the caller and gatekeeping for the healthcare organization (Holmström \& Dall’ Alba 2002), which might create ethical dilemmas (Lännerström et al. 2013). Greenberg (2009) has proposed a commonly used model of the process of telephone nursing. It includes the three phases of information gathering, cognitive processing and output. To adequately perform their work, telephone nurses are dependent on a well‐functioning organization (Ström et al. 2006). There are several aspects inherent to telephone nursing that might hamper patient safety, related to the surrounding society, the organization, the telephone nurse and to the caller. The latter includes caller who are angry, demanding or non‐native speakers (Röing et al. 2013). \cite{holmstrom2017frequent}

Our research concerns aberrant cases that represent frequent attender (FA) patients. While outliers by their nature represent a small subsection of population morbidity, the significance in the detection of FA cases relates to the ultimately disproportionate impact they pose on healthcare provision. Moreover, the aspects which constitutes an FA case may potentially present prodromal features of chronic diseases.   \cite{holmstrom2017frequent}  

Several recent studies have attempted to characterise frequent callers to the ambulance service with regards to chief complaint at the time of patient presentation. However, frequent callers may present with multiple complaints on each occasion that they call 999 and little is known about the underlying causes of frequent calls to the ambulance service. The evidence regarding the best way to manage this patient group is confined to one pilot study limited by a very small sample size.\cite{edwards2015frequent}


ACCORDING TO \cite{scott2014describing}
who analysed specifically frequent callers of emergency ambulance services (and not OOHC GP cooperatives, although were very likely to call during OOHC periods)

>Frequent callers displayed many different characteristics compared with the population, and there appeared to be different subgroups of frequent callers based broadly on their primary medical diagnosis or psychiatric classification. These subgroups appear to be similar to those identified in EDs,2–4 although further research is required to determine where similarities and differences may exist.

>Frequent callers were more likely to be assigned ‘abdominal pain/problems’, ‘breathing problems’, ‘chest pain (non-traumatic)’, ‘headache’, ‘psychiatric/abnormal behaviour/suicide attempt’ and ‘sick person’ call codes. These were identified primarily via the population comparison, and with the exception of ‘headache’ were triangulated with the multiple regression analysis. The lack of the headache call code in the multiple regression analysis suggests that although headache is a more common call code for frequent callers, it had little impact upon the total number of calls made, and is assigned sporadically by the call taker, most likely as a result of headache calls being combined with a more serious complaint which takes priority.





%> The following paragraph should go elsewhere



Data from currently available studies on presentation rates for FAs in
OOHC is variable. FAs typically comprise between 7.7\% and 20.1\% of OOHC patients depending on the setting and country.\cite{leutgeb2018patients} Sociodemographic characteristics such as social deprivation or low income are factors that have been linked to frequent attendance in OOHC. Infants and elderly patients are also more likely to be FAs in OOHC. FAs are much more likely than the general population to suffer from complex co-morbidities, including ill-defined long term physical conditions. \cite{patel2015clinical} 

In addition, in some studies, patients with chronic diseases and those suffering from psychiatric disorders, have been identified as frequent or very FAs both in primary care in general and in OOHC care (though such morbidity may not be apparent in individual incidents of care).\cite{ng2015frequent} Moreover, FAs with chronic diseases presenting at an OOHC-center often additionally suffer from major depression. Such additional mental disorders are underreported in OOHC.\cite{bhroin2019profiling} These factors can make the early identification of probable FA cases a useful avenue for medical intervention, and potentially provide treatment more suitable than ad-hoc telenursing and triage.





 

Sometimes known as Electronic Medical Record or Electronic Patient Record

The concept of an Electronic Health Record (EHR) was set out in order to close the gap between institution-specific patient data and a comprehensive, longitudinal collection of the patient's health data. \cite{quaglio2016health}

EMRs are usually used in the US for patient billing
\cite{alpert2019electronic}


In addition to time spent, EHR documentation is encumbered by other challenges. In the current system, each provider writes his or her own encounter-based notes, leading to redundancy, fragmentation, and lack of a single shared clinical narrative. This problem is further aggravated when a patient receives care across organizations whose EHRs are not interoperable. Redundant documentation \cite{warner2019s}


Charting increasingly began to reflect institutional priorities in response to changes in social contexts of practice, which could only be achieved by the standardisation of documentation practices using integrated computer networks. The result has been a de‐emphasis on the patient's narrative as a source of input into the health record, accompanied by a shift towards representing the patient as a set of data points or metrics.
\cite{doi:10.1111/nup.12112}

We identified 3 EHR challenges common among practices integrating care. First, practices hired new types of clinicians, such as psychologists in primary care practices and nurse practitioners in community mental health centres, who generated data not previously documented or tracked by existing EHR systems (e.g., patient health questionnaire  scores, behavioural health visit notes, consultation notes, referrals to outside services). EHRs generally lacked standard templates to document these additional inputs in structured data fields. This limitation made it difficult for practices to find, extract, and track relevant behavioural health and physical health information to monitor quality and improve the delivery of integrated care. Second, integrated teams had specific communication and care coordination needs, such as use of shared care plans to coordinate tasks for patients receiving integrated care services, and reported needing the ability to see when each other's tasks were completed. EHRs typically did not have templates that supported shared care plans for both primary care and behavioural health needs. However, EHRs that had tasking functions were helpful in enabling some types of communication and coordination between team members. Third, EHRs were not interoperable with other EHR systems or with tablet devices used by practices to administer behavioural health screening surveys. This lack of system interoperability created further barriers to document patient encounters, access needed information at the point of care, and easily and consistently communicate information between primary care and BHCs.
\cite{cifuentes2015electronic}

EHR-like systems and e-prescription are priorities in various EU e-Health Action Plan and in the policies of several Member States (respectively, 27 and 22 EU countries) [31, 66]. However, the general political commitment to these e-health fields is at different stages of implementation across countries \cite{stroetmann2011european}

On average 77.4\% of GPs across Europe store patient consultations in some form of EHR.  
\cite{de2015basic}


However, most of the EPRs used in health centers are proprietary systems built with
different architecture, business rules, information technologies and models, in addition to incompatible clinical terminology. These facts hinder the interoperability among the HIS, making it difficult for health professionals to provide adequate care
\cite{gomes2018marcia}


Generally patient records are written by highly skilled physicians and nurses
using domain specific terms. For example, patient record text is very domain specific
depending on which medical discipline it is written in. Each discipline or domain
within medicine uses its own set of terms that can be incomprehensible by other
disciplines.
\cite{dalianis2018characteristics}

The patient records are written under time pressure; the patient record systems
do not contain any spelling correction (or grammar checking) system due to the
difficulties of building such a function because of the complicated non-standard
vocabulary used within health care.
\cite{dalianis2018characteristics}

Patient records are primarily written for hospital internal use and for mnemonic
reasons. Daily running notes might contain more spelling errors or noisiness than
discharge letters that are read by a larger audience 
\cite{ehrentraut2012detection}

