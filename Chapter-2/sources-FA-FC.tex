billings2013dispelling
>>>>>>>>>>>>>>>>>>>>>>>>>>>>>>>>>>>>>>>>>>>>>
A growing body of research has begun to disprove much of the legend, at least among readers of academic emergency medicine literature. There is no standard definition for frequent ED use,8–10 and it is not clear that having specific cutoffs (in terms of number of visits) is useful from a policy standpoint.11 However, many studies have defined frequent use as three to five ED visits in a year.

Compared to occasional ED users, frequent users have been found to have high illness acuity levels (often with multiple chronic conditions), high rates of primary and specialty care use outside the ED, and frequent hospital admissions for serious conditions

Previous studies have sought to characterize the heterogeneous population of frequent ED users. Nonetheless, few studies have examined ED use and costs in the context of the use of other health services over multiple years or attempted to develop models to predict future frequent ED use to assist with targeting interventions for high-risk patients.

Identifying which patients to target for interventions to improve their health outcomes and reduce their frequent use of health services has proved challenging. Therefore, we conducted predictive modeling to determine whether patients at various levels of ED use in the index year could be identified.

Behavioral health problems were also prevalent, with 58.8 percent of the patients having a history of substance use, 72.3 percent a history of mental illness, and 48.9 percent a history of both substance use and mental illness. ED use for substance use and mental illness was substantially higher among those identified by predictive modeling than for all ED patients. However, only 10.2 percent of ED visits during the index year had a principal diagnosis of substance use, and the figure for mental health diagnoses was 5.6 percent.

Patients identified by the model were also at higher risk of becoming serial users, with three or more or five or more visits in the index year and each of the following two years. This phenomenon was even more pronounced in the predictive modeling used to identify patients at risk of higher numbers of ED visits in the index year.

It is difficult to assess the appropriateness or necessity of ED visits from administrative data alone. However, rates of ED use for nonemergent and emergent but primary care treatable visits24 were comparable among all levels of ED frequency. Again, contrary to the urban legend, most repeat ED users in this study did appear to have relatively strong linkage to ambulatory care, at least as evidenced by their high rates of primary and specialty care visits. Except for ED users with ten or more visits in the index year, ambulatory care visit rates actually exceeded ED visit rates.

It is also important to note that only a small number of “frequent fliers” are ultra-high ED users or serial high ED users, with frequent ED use year after year. To date, most thinking by providers and policy makers about the problem of frequent ED users has focused on these serial users, but the overwhelming majority of frequent users have only episodic periods of high ED use, instead of consistent use over multiple years. More needs to be learned about these patients as well (they, too, could be interviewed in the ED), and predictive modeling and quick intervention will probably be critical since their repeat ED use is unlikely to continue over time.

CLASSIFICATION
Patients' prior diagnostic history was based on International Classification of Diseases, Ninth Revision, Clinical Modification (ICD-9-CM), codes from all diagnosis fields (that is, both principal and secondary diagnoses) from all inpatient and outpatient records that included diagnostic information. The Charlson Comorbidity Index was calculated for each patient as one measure of the acuity of a condition or severity of a disease
The strongest results were obtained in predicting patients who would have three or more visits (including the index visit) during the index year, with a positive predictive value of 0.663, sensitivity of 0.229, and specificity of 0.952. Comparable results were obtained for predicting five or more, eight or more, and ten or more visits, although with lower sensitivity.

<<<<<<<<<<<<<<<<<<<<<<<<<<<<<<<<<<<<<<<<<<<<<



weber2012defining
>>>>>>>>>>>>>>>>>>>>>>>>>>>>>>>>>>>>>>>>>>>>>
According to these findings, it might be tempting to redefine frequent use according to the “breakpoint” in patient characteristics. However, the breakpoints are not clear cut, and they overlap. Mental illness is present in the group with 1 to 6 visits; the proportion of patients with many primary care visits gradually increases from 1 to 18 visits. The characteristics of highly frequent users change at 15 visits according to their residence and at 17 visits according to age, and the number of primary care visits is about the same. These results agree with previous research showing that the factors prompting frequent use exist along a continuum, and either a mathematical or descriptive definition of a frequent user is an oversimplification. Patients with higher numbers of visits are more likely to have certain characteristics, but it would be inaccurate and unfair to say that after a certain number of visits, the aforementioned stereotypes of the frequent user necessarily apply. We must also be careful not to place value judgments on these different tiers of use because of their statistical association with less “desirable” characteristics. It would be ludicrous to say that 6 visits a year to an ED are acceptable but 7 are not.

Frequent use is only “too frequent” if the patient could have been better served in another setting or with another approach. Instead of trying to figure out how many visits are too many, we should be examining the reasons for the visits and whether patients are receiving what they need. Doupe et al5 have also done a great service in clarifying this point. Setting aside the exact cutoffs of frequent and highly frequent use, Doupe at al5 have demonstrated that policies need to be aimed at the causes. Interventions for patients whose frequent use is due to their chronic illness will likely involve improved access to primary care and more intense monitoring of chronic conditions, whereas those whose visits are more likely related to poverty, substance abuse, and mental illness will need intense social interventions. However, because only a small percentage of patients with frequent use continue that pattern after a few years, identifying those who will benefit most from these interventions is essential.4, 6

<<<<<<<<<<<<<<<<<<<<<<<<<<<<<<<<<<<<<<<<<<<<<



doupe2012frequent
>>>>>>>>>>>>>>>>>>>>>>>>>>>>>>>>>>>>>>>>>>>>>
To develop a frequent ED use threshold, select patient characteristics were viewed by ED visit frequency, and “breakpoints” in these trends were sought (Figure 2). No breakpoints were observed at 4 ED visits, suggesting that these patients are not unique.


Using data from Figure 2, we also defined a subset of patients with greater than or equal to 18 ED visits (ie, highly frequent users). Compared with others, fewer of these patients were aged 65 years or older, whereas more had mental health issues and lived in a core area. These patients composed a small portion (0.2\%; N=223) of all users but a more sizable proportion (3.6\%; N=7,177) of ED visits. Overall, frequent users (7 to 17 visits) composed 2.1\% of users and 9.9\% of ED visits, whereas highly frequent users (≥18 visits) composed 0.2\% and 3.6\% of users and visits, respectively 

Compared with less frequent users, a greater proportion of frequent ED users were aged 75 years or older and lived in either a core or low-income area (Table 2). Also, many more frequent versus less frequent users had chronic diseases such as asthma, diabetes, and ischemic heart disease. Similarly, a greater proportion of frequent versus less frequent users had mental health issues.

In some instances, highly frequent ED users had characteristics that were similar to, but more pronounced than, those of frequent users (Table 2). For example, about 57\% of highly frequent users lived in a core or lowest-income area, and 67.3\% of these residents received a diagnosis of substance abuse. Also, 70.0\% of highly frequent users had 7 or more ED visits in the previous calendar year.

In other instances, highly frequent and frequent ED users differed markedly (Table 2). For example, 31.4\% of frequent ED users were aged 65 years or older versus only 9.8\% of highly frequent users. Also, with the exception of arthritis and to a lesser extent asthma, fewer highly frequent versus frequent ED users had physical diseases. Although highly frequent users also had many other health care contacts (14,311 as counted for frequent users; median of 1.0 contacts per week; data not shown), unlike frequent users this trend was provider specific. For example, only marginally more highly frequent versus frequent users had 7 or more primary care physician visits, and fewer highly frequent users had 3 or more specialist physician visits. Conversely, a greater proportion of highly frequent versus frequent users consulted 3 or more primary care physicians during the study period, and 20.2\% (versus 13.0\% of frequent users) called Health Links 7 or more times. Last, fewer highly frequent versus frequent users died within 180 days of their index ED visit.

During descriptive comparisons, we noticed a high degree of colinearity between select mental illnesses. For example, at least 70\% of frequent and highly frequent users with another mental health issue also had depression, and highly frequent users with other mental health issues often had anxiety. 

Other results, however, show that highly frequent users may be unique. For example, despite having more ED visits, the adjusted odds of having 7 or more physician visits and 3 or more hospital admissions were similar for frequent and highly frequent users, and these patients had a similar risk of dying after their last (index) ED visit. Last, with the exception of arthritis, the adjusted odds of having a physical disease were similar for highly frequent versus frequent ED users.

Fourth, contrary to our findings, others have shown that sociodemographic measures strongly influence ED use.9, 24 As a general rule, these factors are captured poorly in administrative records.

<<<<<<<<<<<<<<<<<<<<<<<<<<<<<<<<<<<<<<<<<<<<<




krieg2016individual
>>>>>>>>>>>>>>>>>>>>>>>>>>>>>>>>>>>>>>>>>>>>>
Frequent ED visits represent substantial costs to the health care system [12, 13, 14]. They also decrease ED efficiency [15], contribute to ED overcrowding [16, 17] and can potentially impact services by redirecting them away from urgent cases [13]. Quality of care received may be suboptimal for frequent ED users, as care can be fragmented, episodic and poorly coordinated [18, 19, 20, 21]. Physicians could also hold biases and feel less empathy for frequent ED patients [22]. Hence, the use of ED services by frequent users can often be perceived as inappropriate and nonurgent [3, 23]. As a result, the uncoordinated acute care received in the ED by these patients can be less effective compared to what they receive or would receive in primary care [24, 25, 26]. Previous studies have shown that frequent ED users are more likely to have chronic diseases, suffer from mental illnesses or have substance use disorders [4, 5, 27, 28, 29, 30, 31, 32]. It has also been observed that from 1 year to the next, there is a natural decline of ED use by frequent users. However, the attrition rates of those who remained frequent ED users over the years decreased [30], making them an ideal group for targeted interventions.Improving our understanding of the needs of frequent and chronic frequent ED users and defining a new approach to correctly identify these patients would allow healthcare professionals to intervene before frequent use occurs and redirect them to more appropriate health services [33].

There was no standard definition of a frequent user, although the majority of the included studies (8) defined frequent use as 4 or more ED visits during a 12-month period [4, 5, 6, 7, 8, 9, 10, 11]. Other definitions varied between 3 [39] and 17 [27] or more ED visits per year. A total of eight studies analyzed frequent ED use during a multiple-year period [7, 9, 30, 36, 38, 39, 40, 42]. However, only four [7, 36, 38, 39] of these studies used regression modeling to identify factors predicting chronic frequent ED use. 

Andrén KG, Rosenqvist U. Heavy users of an emergency department—A two year follow-up study 1987
Billings J, Raven MC. Dispelling an urban legend: frequent emergency department users have substantial burden of disease. Health Aff. 2013;32
Rask KJ, Williams MV, McNagny SE, Parker RM, Baker DW. Ambulatory health care use by patients in a public hospital emergency department. J Gen Intern Med. 1998
Okuyemi KS, Frey B. Describing and predicting frequent users of an emergency department. J Assoc Acad Minor Phys. 2001


Three studies [27, 30, 37] found that males were more likely to be frequent users. Another study [11] found that being a female was predictive of frequent ED attendance and that male patients had a lower likelihood of being frequent attenders. A total of four other studies [3, 9, 33, 42] also analyzed gender as a potential independent factor but found that it was not a significant predictor of frequent ED use.
In a population of 59,803 patients, patients in the geriatric age group (75 and older) were more likely to be frequent ED users compared to younger adults (20 to 49 years old). They also found that, compared to young adults, the odds of frequent use were lower for older adults (50 to 74 years old) [11]. Another study [37] also found that being 75 years old or more was a predictive factor of frequent ED use. However, two other studies stated otherwise as they found that frequent ED users were more likely to be between 30 and 59 years old and significantly younger than the non-ED users [30, 41]. Conversely, six studies stated that patient age was not a significant risk factor of frequent ED use [3, 9, 10, 27, 33, 42].


Frequent ED users tended to heavily use other medical services; some of the variables used to describe their health care use included multiple visits to a specialist physician [9, 27, 41] multiple visits to a primary care provider [5, 8, 27, 40], calling health helplines, [27] being previously hospitalized [3, 6, 8, 10, 41], having outpatient visits [4, 10], visiting a clinic [6, 10], having a history of past ED use [39],

Multiple studies stated that physical diseases were an important contributing factor in heavy ED use [3, 4, 8, 10, 27, 30, 37, 40]. Frequent ED users suffered from various medical conditions which included chronic diseases [10, 27], pulmonary diseases [3, 8, 10, 37] , respiratory diseases , cardiovascular diseases. ED patients with mental illnesses were also at higher risk of becoming frequent ED users [4, 6, 8, 9, 10, 27, 30, 42].  


<<<<<<<<<<<<<<<<<<<<<<<<<<<<<<<<<<<<<<<<<<<<<



kuek2019characteristics
>>>>>>>>>>>>>>>>>>>>>>>>>>>>>>>>>>>>>>>>>>>>>
(3–5), Canada (6,7), Australia (8), and Korea (9), to name but a few. Frequent attenders to EDs further compound the problem  by  taking  up  disproportionately  large amount of resources; a Singapore study revealed frequent attenders comprising of only 7.8\% of total ED patients contributed to 26.4\% of all attendances (10).




<<<<<<<<<<<<<<<<<<<<<<<<<<<<<<<<<<<<<<<<<<<<<



middleton2017frequent
>>>>>>>>>>>>>>>>>>>>>>>>>>>>>>>>>>>>>>>>>>>>>



<<<<<<<<<<<<<<<<<<<<<<<<<<<<<<<<<<<<<<<<<<<<<


tarnqvist2017scene
>>>>>>>>>>>>>>>>>>>>>>>>>>>>>>>>>>>>>>>>>>>>>



<<<<<<<<<<<<<<<<<<<<<<<<<<<<<<<<<<<<<<<<<<<<<


pirkis2016frequent
>>>>>>>>>>>>>>>>>>>>>>>>>>>>>>>>>>>>>>>>>>>>>

LIFELINE>
20 times a month, (telephone helpline)
DIAMOND>
one a week or more (Diagnosis, Management and Outcomes of Depression in Primary Care (diamond) study)

LIFELINE >
Suicidality, self-harm, mental health problems and issues related to crime, child protection and domestic violence were all associated with being a frequent caller.
 In addition, the interviews pointed to some themes that were common to all callers. These drove their frequent use of the service and included: positive reinforcement; social isolation; anonymity; and unrestricted access.

DIAMOND>
everal indicators of social isolation (e.g., living alone and being bothered a lot by not having a confidant). Several physical health factors were also predictive of frequent use (e.g., having a chronic disease and/or self-rating of own health as poor or fair), as were a number of mental health factors (e.g., having anxiety, major depression, a likely personality disorder and/or suicidal thoughts, and/or using antipsychotic medication). Frequent use of telephone helplines was also associated with using emergency departments, psychologists and psychiatrists. It was also associated with an increased likelihood of visits to more than one general practitioner (GP). In addition, frequent use was associated with greater levels of dissatisfaction with access to health services.


Several key findings stand out. Frequent callers are relatively few in number but they account for a substantial proportion of calls. They have a heavy reliance on helplines, perhaps because they are isolated and have relatively few social supports. They are by no means just “time wasters”, however; they have high levels of need, as evidenced by the fact that they have major mental health problems (including anxiety, depression and suicidality) and are often in crisis. They also make use of other services for their mental health problems, including GPs, allied health professionals (e.g., psychologists), psychiatrists and emergency departments. Current service models are not meeting their needs and are reinforcing their calling behaviour.

Frequent callers would be allocated to one of these specialised TCSs who would develop a rapport with them, establish rules about the timing and duration of their calls, and help them work towards clearly defined goals. The caller and the TCS would reach an agreement about how often the caller could use the service, the type of care he or she should expect to receive, and what to do in the case of an emergency. The TCSs would provide a more intensive, high level of counselling than the standard telephone helpline service.

The model recognises that frequent callers are likely to already be using a range of these other services, including GPs and mental health specialists. It is not about creating new linkages but improving the quality of existing ones, reducing reliance on multiple providers (e.g., several GPs), and fostering consistent approaches. For example, there would be instances in which the TCS might work with the caller and his/her GP on a shared care plan.


<<<<<<<<<<<<<<<<<<<<<<<<<<<<<<<<<<<<<<<<<<<<<


middleton2016experiences
>>>>>>>>>>>>>>>>>>>>>>>>>>>>>>>>>>>>>>>>>>>>>

reported calling 20 times or more in the past month 

Each call is treated as a unique encounter 

Firstly, respondents called seeking someone to talk to, help with their mental health, and assistance with past and present negative life events. Secondly, respondents indicated that the response they received from Lifeline helped to meet their short-term needs and the ability to call regularly without restrictions was favourably viewed. Thirdly, respondents described three distinctive calling behaviours which were classified as reactive, support-seeking and dependent. These calling behaviours appeared to develop over time and were closely related to respondents’ reasons for calling and the response they received.

Frequent users call about issues that are ongoing in their lives. However, the current model of care makes it difficult for crisis helpline staff to adequately address frequent users’ needs in a single encounter. Furthermore, frequent users are sometimes dissatisfied with the response they receive and call back looking for a TCS with a different approach. This mismatch between the model of care offered by crisis helplines and the support frequent users are seeking suggests that they may benefit from more continuous care over a period of time.


<<<<<<<<<<<<<<<<<<<<<<<<<<<<<<<<<<<<<<<<<<<<<


middleton2016health
>>>>>>>>>>>>>>>>>>>>>>>>>>>>>>>>>>>>>>>>>>>>>

The complex health needs of frequent users mean that they are likely to require a range of healthcare services, yet their health service use patterns are poorly understood. 

However, the evidence indicates that frequent users are making use of other parts of the healthcare system. Several studies have found that frequent users commonly report receiving current or prior treatment from professionals who provide mental healthcare in a range of settings (Bartholomew and Olijnyk 1973; Bassilios et al. 2015; Burgess et al. 2008; Farberow et al. 1966; Greer 1976; Lester and Brockopp 1970; Sawyer and Jameton 1979). Our own previous study in this area shows that repeat callers of telephone helplines are more likely to have received mental healthcare from a GP in the past year than those who called only once or did not call at all (Bassilios et al. 2015). However, whether these healthcare services are able to meet the needs of frequent users of telephone helplines requires further investigation.

Frequent use of telephone helplines was associated with younger age (18–34 years), living alone, difficulties managing on available income and being bothered a lot by not having a confidant. The odds of reporting frequent use compared to non-frequent use and no use of telephone helplines was also higher for those who had a chronic disease and/or rated their health as poor or fair. In relation to mental health factors, anxiety at baseline, major depressive syndrome, a high probability of personality disorder, suicidal thoughts and the use of antipsychotic medications was strongly associated with frequent use of telephone helplines.

Frequent use of telephone helplines was also associated with visits to a psychologist, psychiatrist, and/or the emergency department in the previous 3 months. An association was also found between the number of GPs visited in the 3 month period and the likelihood of being a frequent user of telephone helplines.

The frequent use of telephone helplines by primary care attendees with depressive symptoms does not appear to be driven by a lack of access to face-to-face healthcare services. These individuals access a larger range of healthcare services in comparison to non-frequent and non-users of telephone helplines, but their more complex health needs appear to remain unmet. The development of a model of care that aims to better meet the needs of these users is required. This would require understanding whether GPs and other mental health services find it difficult to manage these patients and the type of support frequent users perceive telephone helplines provide them.

<<<<<<<<<<<<<<<<<<<<<<<<<<<<<<<<<<<<<<<<<<<<<


holmstrom2017frequent
>>>>>>>>>>>>>>>>>>>>>>>>>>>>>>>>>>>>>>>>>>>>>
A frequent caller was mainly defined as: ‘a person calling once or more per week or as someone who made repeated or persistent calls’.

Some of the frequent callers were also frequents attenders, as they combined calling frequently with showing up unannounced at the primary healthcare centre.

The telephone nurses believed that the underlying reasons for the calls were related to the callers’ ‘worries and fears’, ‘physical symptoms’ or ‘social aspects’.

An increased work load was commonly experienced due to frequent callers. These calls were experienced as ‘stressful and time consuming’ to handle, creating a time pressure. Moreover, the calls were described as ‘being frustrating’ and the nurses said that they ‘lacked the potential to help’ frequent callers.

The patients were often instructed by hospital staff to turn to the primary healthcare centre. However, when calling the telephone nurses, patients risked being referred back to the hospital, as their problems were too complicated to be managed in primary care.

These calls were challenging because the underlying causes of the calls were often complex. Strategies to provide better care and increase access for other callers should be decided on an organizational level and not left to the individual RN to handle.


<<<<<<<<<<<<<<<<<<<<<<<<<<<<<<<<<<<<<<<<<<<<<


spittal2015frequent
>>>>>>>>>>>>>>>>>>>>>>>>>>>>>>>>>>>>>>>>>>>>>

respondents were categorised into three groups: one-off users (once/just today); episodic users (2–19 times); and frequent users (≥20 times). These categories were selected as
they reflect Lifeline’s current classification of users; in particular, Lifeline defines a frequent user as someone who calls the service 20 times or more in a month (Spittal et al. 2015). 

This pattern of service use differs to the current model of care offered by crisis helplines which is designed to provide one-off support. Instead, frequent users may be better suited to a model of care that is designed to provide ongoing support. If such a model were offered, a ‘whole system approach’ would be suitable as previous research has shown that frequent users already receive mental healthcare from a general practitioner (Bassilios et al. 2015, Middleton et al. 2016) and mental health professionals (Burgess et al. 2008, Middleton et al. 2016). Such an approach would involve integrating clinical and professional services with an ongoing non-clinical service that specifically addresses the need for emotional support and occasional crisis support. This would first require a better understanding of the needs of frequent users and why they are seeking ongoing emotional support from crisis helplines (Finch et al. 2008). T


<<<<<<<<<<<<<<<<<<<<<<<<<<<<<<<<<<<<<<<<<<<<<


scott2015delivering
>>>>>>>>>>>>>>>>>>>>>>>>>>>>>>>>>>>>>>>>>>>>>



<<<<<<<<<<<<<<<<<<<<<<<<<<<<<<<<<<<<<<<<<<<<<


scott2014describing
>>>>>>>>>>>>>>>>>>>>>>>>>>>>>>>>>>>>>>>>>>>>>



<<<<<<<<<<<<<<<<<<<<<<<<<<<<<<<<<<<<<<<<<<<<<


edwards2015frequent
>>>>>>>>>>>>>>>>>>>>>>>>>>>>>>>>>>>>>>>>>>>>>



<<<<<<<<<<<<<<<<<<<<<<<<<<<<<<<<<<<<<<<<<<<<<




In the United States health insurance had a strong association with frequent attenders in emergency departments,\cite{lacalle2010frequent} in a way which would not be replicated in Ireland due to varying models in healthcare delivery.\cite{van2000equity} [betterbib] 












>>>>>>>>>>>>>>>>>>>>>>>>>>>>>>>>>>>>>>>>>>>>>
INGER \cite{holmstrom2017frequent}
(FC) - Sweden

Telephone nursing is expanding in many countries and is often recommended as the citizens’ first level of care.

The telephone nurses described a frequent caller as a person calling once a week or more.

The Swedish Healthcare Direct 1177 (SHD) is a telephone nursing service similar to the NHS 111 in the UK, LINK in Canada and Health Direct in Australia (ELSEWHERE)

paid to frequent callers, the reasons behind these calls and the management strategies applied by Registered Nurses (RNs)




>>>>>>>>>>>>>>>>>>>>>>>>>>>>>>>>>>>>>>>>>>>>>


